\documentclass{novelette}
\title{Novelette}
\author{Novelette}
\mode{final}
\intent{SWOP}
\begin{document}
\mainmatter
\begin{opening}
\null\null\null
\name{NOVELETTE}
\null
\desc{LuaLaTeX Document Class}
\null\null
\end{opening}
\init{D}{ocumentation} is \medcap{HTML}:
doc/lualatex/novelette/novelette.html\par
\scene{-}
Novelette is intended for one thing only: Popular fiction, printed to paper
(not Ebook), black ink, probably using print-on-demand. If you wish to write
a detective novel or science fiction, you are in the right place. But if
you wish to write an academic book, or graphic novel, or anything involving
color interior, you are in the wrong place.

To keep you focused on writing, rather than on LaTeX coding, Novelette offers
a very limited command set. For everything you need to do, there is a simple
way do to it, using understandable commands, and useful defaults. For anything
you do not need to do, it is quite possible that it cannot be done.

You can place black/white raster images, but not vector diagrams. You can
create chapters with uniform style, but Novelette will not automatically
index them for you.

A distinguishing feature is Novelette's use of fixed line grid. The result
looks and reads the way it should, for popular fiction.

The \medcap{HTML} documentation tells you everything you need to know,
and provides examples of how to do common tasks. Read it. But \ital{do not}
read standard LaTeX how-to manuals or online help. Those apply to other
document classes, not to Novelette.

If you use a print service requiring \medcap{PDF/X}-1a:2001 files, Novelette
can do it. The capability is built-in. All you need to do is request it in the
settings, and process images according to specs.

\scene{Frequently Asked Questions}
\begingroup\setlength\parindent{0pt}
\ital{Q.} Who should use Novelette?\\
\ital{A.} Anyone who knows that TeX should not rhyme with tech.\par
\ital{Q.} Why did you create Novelette?\\
\ital{A.} I live on a California ridge overlooking the ocean. That is,
I could see the ocean, except numerous tall trees are in the way. I am hoping
that Novelette will revive printed books, so that the trees are cut down
to make paper. None of that namby-pamby concern for literacy.\par
\ital{Q.} How did you program Novelette?\\
\ital{A.} I obtained a hundred feral cats from the rescue center, and a
hundred old computers from the recycler. I let the cats walk across the
keyboards until Novelette emerged. Be assured that it is more intelligible
than most dissertations written with TeX.\par
\ital{Q.} Why did you disable all of the normal methods for doing math?\\
\ital{A.} Jealousy. The cats know more math than I do.\par
\ital{Q.} What is the best way to fix catcodes?\\
\ital{A.} Take the cats to a veterinarian.\par
\ital{Q.} Why doesn't Novelette use more expl3 syntax?\\
\ital{A.} I only program expl3 when the waning crescent Moon is in the Hyades,
at the Spring Equinox. Didn't happen this year. Sorry.\par
\ital{Q.} My uncle is a major celebrity. Can I make millions of dollars
by using Novelette to write a tell-all book about him?\\
\ital{A.} You can probably make the money, but please don't use Novelette.\par
\ital{Q.} My primal scream therapist suggested that I should write my
memoir. Would Novelette be good for that?\\
\ital{A.} Maybe, but the bookshelves are already full of that genre.\par
\ital{Q.} How can I put a Feynman diagram into Novelette?\\
\ital{A.} Use a sledgehammer. The bigger, the better.\par
\ital{Q.} They are everywhere. Tittering in the night. Glaring at me with
their beady little eyes. Barely concealing their thoughts that someday,
everything will be theirs. \ital{What can I do?}\\
\ital{A.} If you are talking about your grandchildren, just be nice to them.
If you are writing science fiction, try Novelette.\par
\endgroup

\nocleartoendtrue
\end{document}
