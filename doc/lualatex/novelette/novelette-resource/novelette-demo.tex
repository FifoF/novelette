% !TeX TS-program = lualatex
% !TeX encoding = UTF-8
\documentclass{novelette}
\title{Selected Readings of Mark Twain}
\author{Samuel Clemens}
\mode{final}
\intent{SWOP}
\setlang{en-US}
\guides{9,18}
\begin{document}

\frontmatter

\begin{display}
\vfil
\style[scale=1.6]{Selected Readings\\of\\Mark Twain}
\null\null\null\null
\vfil
\end{display}

\memo{Frontispiece image, instead of the usual blank page.}
\begin{display}
\vfil
\image{twainline.png}
\style{Samuel Clemens}
\vfil
\end{display}

\begin{display}
\null\null\null\null\null\null
\style[font=deco,scale=3t]{Selected Readings\\of\\Mark Twain}
\null\null
\style[scale=2]{Samuel Clemens}
\vfill
\style{Demonstration of the Novelette Document Class}
\end{display}

\begin{display}
\vfill
\style{Selected Readings of Mark Twain\\
/ Samuel Clemens}
\style{Mark Twain is the pseudonym of American\\
author Samuel Clemens (1835--1910)}
\style{The text and image are in the Public Domain\\
of the United States of America, and of nations\\
subscribing to the Berne Copyright Convenion,\\
due to passage of time (over 100 years) since\\
initial publication and the death of its author.}
\style{This is a work of fiction.}
\style{Novelette project page:\\
https/github.com/rallg/novelette}
\style{ISBN \lnum{0-00-000000000-0}}
\end{display}

\begin{display}
\vfil
\name{Editor's Note}
\null\null\null\null
\begin{blockindent}[2,2]
This is a demonstration of the Novelette document class, compiled
with the LuaLaTeX typesetting engine.

The words of Samuel Clemens
have been edited for flow and appearance. None of this content
should be used as literary reference.
\end{blockindent}
\vfil
\end{display}

\blankpage

\mainmatter

\begin{opening}
\null\null\null\null\null\null
\name{1. Decay of the Art of Lying}
\null
\desc{Excerpt from an Essay, 1880}
\null\null\null\null\null
\end{opening}

\init{I }{do not mean to suggest} that the \ital{custom} of lying has
suffered any decay or interruption. The Lie, as a Virtue, a
Principle, is eternal; the Lie, as a recreation, a solace, a refuge in
time of need, the fourth Grace, the tenth Muse, man's best and surest
friend, is immortal, and cannot perish from the earth.

My complaint simply concerns the decay of the \ital{art} of lying.
Not one of us can contemplate the lumbering and slovenly lying of the
present day without grieving to see a noble art so prostituted.

No fact is more firmly established than that lying is a necessity of our
circumstances. The deduction that it is then a Virtue goes without
saying. No virtue can reach its highest usefulness without careful and
diligent cultivation. Therefore, it goes without saying that this one
ought to be taught in the public schools---even in the newspapers. What
chance has the ignorant uncultivated liar against the educated expert?
What chance have I against a lawyer? \ital{Judicious} lying
is what the world needs. I sometimes think it were even better and safer
not to lie at all than to lie injudiciously. An awkward, unscientific
lie is often as ineffectual as the truth.

Now let us see what the philosophers say. Note that venerable proverb:
Children and fools \ital{always} speak the truth. The deduction is plain:
adults and wise persons \ital{never} speak it.

\memo{This is an example of a memo. It is like a comment. Nothing within
the memo braces will print. However, unlike the percent comment, anything
after the memo closing brace (on same line of source text) will print.}

Parkman, the historian, says,
``The principle of truth may itself be carried into an absurdity.'' In
another place in the same chapters he says, ``The saying is old that
truth should not be spoken at all times; and those whom a sick
conscience worries into habitual violation of the maxim are imbeciles
and nuisances.''

That is strong language, but true.
None of us could \ital{live} with an habitual truth-teller;
but thank goodness none of us has to. An habitual truth-teller is simply
an impossible creature who does not exist, and never has existed.
Of course there are people who \ital{think} they
never lie, but it is not so; and this ignorance is one of the very
things that shame our so-called civilization. Everybody lies, every day;
every hour; awake; asleep; in dreams; in joy; in mourning; in silence.
Hands, feet, eyes, attitude---all convey deception.

We are liars, every one. Our mere {howdy-do}
is a lie, because we do not care how you did.
To the ordinary inquirer you lie in return; for you make
no conscientious diagnostic of your case, but answer at random, and
usually miss it considerably. If a stranger calls and interrupts
you, you say with your hearty tongue, ``I'm glad to see you,'' and say
with your heartier soul, ``I wish you were with the cannibals and it was
dinner-time.'' But you did no harm, for you did not
deceive anybody nor inflict any hurt, whereas the truth would have made
you both unhappy.

I think that all this courteous lying is a sweet and loving art, and
should be cultivated. The highest perfection of politeness is only a
beautiful edifice, built, from the base to the dome, of graceful and
gilded forms of charitable and unselfish lying.

What I bemoan is the growing prevalence of the brutal truth. Let us do
what we can to eradicate it. An injurious truth has no merit over an
injurious lie. Neither should ever be uttered. Whoever speaks an
injurious truth in fear of damnation for lying, should
reflect that that sort of a soul is not strictly worth saving.
Whoever tells a lie to help a poor devil out of trouble, is one of whom the
angels doubtless say, ``Lo, let us exalt this magnanimous liar.''

An injurious lie is an uncommendable thing; and so, also, and in the same
degree, is an injurious truth. Lying is universal: we \ital{all} do it.
Therefore, the wise thing is for us
diligently to train ourselves to lie thoughtfully, judiciously; to lie
with a good object, and not an evil one; to lie for others' advantage,
and not our own; to lie healingly, charitably, humanely, not cruelly,
hurtfully, maliciously; to lie gracefully and graciously, not awkwardly
and clumsily; to lie firmly, frankly, squarely, with head erect, not
haltingly, tortuously, with pusillanimous mien, as being ashamed of our
high calling. Then shall we be rid of the rank and pestilent truth that
is rotting the land; then shall we be great and good and beautiful, and
worthy dwellers in a world where even benign Nature habitually lies,
except when she promises execrable weather.

Joking aside, I think there is much need of wise examination into what
sorts of lies are best and wholesomest to be indulged, seeing we \ital{must}
all lie and we \ital{do} all lie.




\begin{opening}
\null\null\null\null\null\null
\name{2. Cooper's Literary Offences}
\null
\desc{Excerpt from an Essay, 1895}
\footnote{James Fenimore Cooper was
a famous romantic novelist of the American Midwest pioneer era. ---Ed.}

\null\null\null\null
\end{opening}


\init[-.1]{T}{here are rules} governing literary art in romantic fiction.

1. They require that a tale shall accomplish something and arrive somewhere.

2. They require that the episodes of a tale shall be necessary parts of
the tale, and shall help to develop it.

3. They require that the personages in a tale shall be alive, except in
the case of corpses, and that always the reader shall be able to tell
the corpses from the others.

4. They require that the personages in a tale, both dead and alive,
shall exhibit a sufficient excuse for being there.

5. They require that when the personages of a tale deal in conversation,
the talk shall sound like human talk, and be talk such as human
beings would be likely to talk in the given circumstances, and have
a discoverable meaning, also a discoverable purpose, and a show of
relevancy, and remain in the neighborhood of the subject in hand, and
be interesting to the reader, and help out the tale, and stop when the
people cannot think of anything more to say.

6. They require that when the author describes the character of a
personage in his tale, the conduct and conversation of that personage
shall justify said description.

7. They require that when a personage talks like a moneyed college graduate
in the beginning of a paragraph, he shall not talk like an uneducated workman
in the end of it.

8. They require that crass stupidities shall not be played upon the
reader, by either the author or the people in the tale.

9. They require that the personages of a tale shall confine themselves
to possibilities and let miracles alone; or, if they venture a miracle,
the author must so plausibly set it forth as to make it look possible
and reasonable.

10. They require that the author shall make the reader feel a deep
interest in the personages of his tale and in their fate.

11. They require that the characters in a tale shall be so clearly
defined that the reader can tell beforehand what each will do in a given
emergency.


Cooper's gift in the way of invention was not a rich endowment; but
such as it was he liked to work it, he was pleased with the effects,
and indeed he did some quite sweet things with it. In his little box of
stage properties he kept six or eight cunning devices, tricks, artifices
for his savages and woodsmen to deceive and circumvent each other with,
and he was never so happy as when he was working these innocent things
and seeing them go.

A favorite one was to make a moccasined person tread
in the tracks of the moccasined enemy, and thus hide his own trail.
Cooper wore out barrels and barrels of moccasins in working that trick.

Another stage-property that he pulled out of his box pretty frequently
was his broken twig. He prized his broken twig above all the rest of his
effects, and worked it the hardest. It is a restful chapter in any book
of his when somebody doesn't step on a dry twig. Every time a Cooper person is
in peril, and absolute silence is worth four dollars a minute, he is
sure to step on a dry twig. There may be a hundred handier things to
step on, but that wouldn't satisfy Cooper, who requires him to turn
out and find a dry twig.

Cooper was a sailor---a naval officer; yet he gravely tells us how
a vessel, driving towards a lee shore in a gale, is steered for a
particular spot by her skipper because he knows of an undertow there
which will hold her back against the gale and save her. For just pure
woodcraft, or sailorcraft, or whatever it is, isn't that neat?

For several years Cooper was daily in the society of artillery, and he ought
to have noticed that when a cannon-ball strikes the ground it either
buries itself or skips a hundred feet or so; skips again a hundred feet
or so---and so on, till finally it gets tired and rolls. Now in one place
he loses some characters in the edge of a wood
near a plain at night in a fog. They hear a cannonblast, and a
cannon-ball presently comes rolling into the wood and stops at their
feet. The heros strike out promptly and follows the track of that cannon-ball
across the plain through the dense fog and finds the fort.

If Cooper had any real knowledge of Nature's ways of doing
things, he had a most delicate art in concealing the fact. For instance:
one of his experts has lost the trail of a person he is tracking through the
forest. Apparently that trail is hopelessly lost. Neither you nor I
could ever have guessed out the way to find it. The expert was not stumped
for long. He turned a running
stream out of its course, and there, in the slush in its old bed, were
that person's moccasin-tracks. The current did not wash them away, as
it would have done in all other like cases--no, even the eternal laws
of Nature have to vacate when Cooper wants to put up a delicate job of
woodcraft on the reader.

Cooper hadn't any more invention than a horse; and I don't mean a
high-class horse, either; I mean a clothes-horse. It would be very
difficult to find a really clever “situation” in Cooper's books, and
still more difficult to find one of any kind which he has failed to
render absurd by his handling of it.

Cooper's proudest creations in the way of “situations” suffer
from the absence of the observer's gift. His eye was
splendidly inaccurate. Cooper seldom saw anything correctly. He saw
nearly all things as through a glass eye, darkly. Of course a man who
cannot see the commonest little every-day matters accurately is
working at a disadvantage when he is constructing a “situation.”

The conversations in the Cooper books have a curious sound in our
ears. To believe that such talk really ever came out of people's mouths
would be to believe that time was of no value to
a person who thought he had something to say; when it was the custom
to spread a two-minute remark out to ten; when a man's mouth was a
rolling-mill, and busied itself all day long in turning four-foot pigs
of thought into thirty-foot bars of conversational railroad iron by
attenuation; when subjects were seldom faithfully stuck to, but the talk
wandered all around and arrived nowhere; when conversations consisted
mainly of irrelevancies, with here and there a relevancy, a relevancy
with an embarrassed look, as not being able to explain how it got there.

Cooper was certainly not a master in the construction of dialogue.
Inaccurate observation defeated him here as it defeated him in so many
other enterprises of his. He even failed to notice that the man who
talks corrupt English six days in the week must and will talk it on
the seventh, and can't help himself.


Cooper's word-sense was singularly dull. When a person has a poor ear
for music he will flat and sharp right along without knowing it. He
keeps near the tune, but it is not the tune. When a person has a poor
ear for words, the result is a literary flatting and sharping; you
perceive what he is intending to say, but you also perceive that he
doesn't say it. This is Cooper. He was not a word-musician. His ear was
satisfied with the approximate word.


There have been daring people in the world who claimed that Cooper could
write English. Now I feel sure, deep down in my heart,
that Cooper wrote about the poorest English that exists.



\begin{opening}
\null\null\null\null\null\null
\name{3. The Mississippi River}
\null
\desc{Excerpt from \ital{Life on the Mississippi}, 1883}
\null\null\null\null\null
\end{opening}

\init[-.1]{T}{he Mississippi} is well worth reading about. It is not a
commonplace river, but on the contrary is in all ways remarkable.

\scene{+I. As a River}

Considering the
Missouri its main branch, it is the longest river in the world---four
thousand three hundred miles. It seems safe to say that it is also the
crookedest river in the world, since in one part of its journey it uses
up one thousand three hundred miles to cover the same ground that the
crow would fly over in six hundred and seventy-five.

It discharges three
times as much water as the St. Lawrence, twenty-five times as much
as the Rhine, and three hundred and thirty-eight times as much as the
Thames. No other river has so vast a drainage-basin. The
Mississippi receives and carries to the Gulf water from fifty-four
subordinate rivers that are navigable by steamboats, and from some
hundreds that are navigable by flats and keels. The area of its
drainage-basin is as great as the combined areas of England, Wales,
Scotland, Ireland, France, Spain, Portugal, Germany, Austria, Italy,
and Turkey; and almost all this wide region is fertile.

It is a remarkable river in this: that instead of widening toward its
mouth, it grows narrower; grows narrower and deeper. From the junction
of the Ohio to a point half way down to the sea, the width averages a
mile in high water: thence to the sea the width steadily diminishes,
until, at the `Passes,' above the mouth, it is but little over half
a mile. At the junction of the Ohio the Mississippi's depth is
eighty-seven feet; the depth increases gradually, reaching one hundred
and twenty-nine just above the mouth.

The difference in rise and fall is also remarkable---not in the upper,
but in the lower river. The rise is tolerably uniform down to Natchez
(three hundred and sixty miles above the mouth)---about fifty feet.
But at Bayou La Fourche the river rises only twenty-four feet; at New
Orleans only fifteen, and just above the mouth only two and one half.

An article in the New Orleans `Times-Democrat,' based upon reports of
able engineers, states that the river annually empties four hundred and
six million tons of mud into the Gulf of Mexico---which brings to mind
Captain Marryat's rude name for the Mississippi---`the Great Sewer.' This
mud, solidified, would make a mass a mile square and two hundred and
forty-one feet high.

The mud deposit gradually extends the land---but only gradually; it has
extended it not quite a third of a mile in the two hundred years which
have elapsed since the river took its place in history. The belief of
the scientific people is, that the mouth used to be at Baton Rouge,
where the hills cease, and that the two hundred miles of land between
there and the Gulf was built by the river. This gives us the age of that
piece of country, without any trouble at all---one hundred and twenty
thousand years.

The Mississippi is remarkable in still another way---its disposition to
make prodigious jumps by cutting through narrow necks of land, and thus
straightening and shortening itself. More than once it has shortened
itself thirty miles at a single jump! These cut-offs have had curious
effects: they have thrown several river towns out into the rural
districts, and built up sand bars and forests in front of them. The town
of Delta used to be three miles below Vicksburg: a recent cutoff has
radically changed the position, and Delta is now \ital{two miles above}
Vicksburg. Both of these river towns have been retired to the country by that
cut-off.

The Mississippi does not alter its locality by cut-offs alone: it
is always changing its habitat \ital{bodily}---is always moving bodily
\ital{sidewise}. At Hard Times, La., the river is two miles west of the
region it used to occupy. As a result, the original site of that
settlement is not now in Louisiana at all, but on the other side of
the river, in the State of Mississippi. \ital{Nearly the whole of that one
thousand three hundred miles of old Mississippi River which La Salle
floated down in his canoes, two hundred years ago, is good solid dry
ground now}. The river lies to the right of it, in places, and to the
left of it in other places.

Although the Mississippi's mud builds land but slowly, down at the
mouth, where the Gulfs billows interfere with its work, it builds fast
enough in better protected regions higher up: for instance, Prophet's
Island contained one thousand five hundred acres of land thirty years
ago; since then the river has added seven hundred acres to it.

\scene{+II. Early History}

Let us drop the Mississippi's physical history, and say a word about its
historical history---so to speak.

The world and the books are so accustomed to use, and over-use, the word
`new' in connection with our country, that we early get and permanently
retain the impression that there is nothing old about it. We do of
course know that there are several comparatively old dates in American
history, but the mere figures convey to our minds no just idea, no
distinct realization, of the stretch of time which they represent.
To say that De Soto, the first European who ever saw the Mississippi
River, saw it in 1542, is a remark which states a fact without
interpreting it.

The date 1542, standing by itself, means little or nothing to us; but
when one groups a few neighboring historical dates and facts around it,
he adds perspective and color, and then realizes that this is one of the
American dates which is quite respectable for age.

For instance, when the Mississippi was first seen by a European, less
than a quarter of a century had elapsed since the death of Raphael;
the driving out of the Knights-Hospitallers from Rhodes by
the Turks; and the placarding of the Ninety-Five Propositions.
When De Soto took his glimpse of the river,
Michael Angelo's paint was not yet dry on the Last
Judgment in the Sistine Chapel; Elizabeth of England was not yet in her teens;
`Don Quixote' was not yet written; Shakespeare was not yet born;
a hundred long years must
still elapse before Englishmen would hear the name of Oliver Cromwell.

Unquestionably the discovery of the Mississippi is a datable fact which
considerably mellows and modifies the shiny newness of our country, and
gives her a most respectable outside-aspect of rustiness and antiquity.

De Soto merely glimpsed the river, then died and was buried in it by his
priests and soldiers. One would expect the priests and the soldiers
to multiply the river's dimensions by ten---the Spanish custom of the
day---and thus move other adventurers to go at once and explore it. On
the contrary, their narratives when they reached home, did not excite
that amount of curiosity. The Mississippi was left unvisited by Europeans
during a term of years which seems incredible in our energetic days.

One may `sense' the interval to his mind, after a fashion, by dividing it
up in this way: After De Soto glimpsed the river, a fraction short of
a quarter of a century elapsed, and then Shakespeare was born; lived a
trifle more than half a century, then died; and when he had been in his
grave considerably more than half a century, the \ital{second} European saw
the Mississippi. In our day we don't allow a hundred and thirty years to
elapse between glimpses of a marvel. If somebody should discover a creek
in the county next to the one that the North Pole is in, Europe and
America would start fifteen costly expeditions thither: one to explore
the creek, and the other fourteen to hunt for each other.



\cleartoend
\end{document}
